\usepackage{etex}
\reserveinserts{28}

\usepackage[T1]{fontenc}
\usepackage[utf8]{inputenc}
\usepackage[spanish]{babel}
\usepackage{amsmath,amsfonts,amsthm}
\usepackage{algcompatible}
\usepackage{algorithmicx}
\usepackage{longtable}
\usepackage{algorithm}
\usepackage{hyperref}
\usepackage{graphicx}
\usepackage{listings}
\usepackage{booktabs}
\usepackage{enumitem}
\usepackage{multirow}
\usepackage{pstricks}
\usepackage{fourier}
\usepackage{amssymb}
\usepackage{times}

\usepackage[font=footnotesize,labelfont=bf]{caption}

\usepackage[scale=2]{ccicons}
\usepackage{minted}

\usepackage{ragged2e}
\justifying

\mode<presentation> {
	\usetheme{m}
}

\uselanguage{spanish}
\languagepath{spanish}
\deftranslation[to=spanish]{Theorem}{Teorema}
\deftranslation[to=spanish]{Definition}{Definición}

\title[CST]{Computing Selected Topics}

\subtitle[Computing Selected Topics]{Resistencia celular y tiempo de respuesta imunológica en un modelo para la infección del VIH}

\institute[ESCOM]{Instituto Politécnico Nacional\\Escuela Superior de Cómputo}

\author{Ian Yevgeni Hernández Sánchez}

\date{Junio 22, 2015}

\hypersetup{
	colorlinks=true,
	linkcolor=black,
	filecolor=black,
	urlcolor=black,
	citecolor=black
}

\urlstyle{same}


% \headsep = 10pt
% \headheight = 20pt

% \renewcommand{\theequation}{\arabic{equation}}
% \renewcommand{\thetable}{\thechapter.\arabic{table}}

% \setlength\parindent{0pt}

% \pagestyle{fancy}
% \renewcommand{\chaptermark}[1]{%
% 	\markboth{\thechapter.\ #1}{}%
% }

% \titleformat{\chapter}{\Huge\bfseries}{\thechapter}{1em}{\vspace{1cm}}

\floatname{algorithm}{Algoritmo}
\renewcommand{\algorithmicrequire}{\textbf{Entrada:}}
\renewcommand{\algorithmicensure}{\textbf{Salida:}}

% \newtheorem{example}{Nota}[chapter]

% \renewcommand{\headrulewidth}{0pt}
% \newcommand{\horrule}[1]{\rule{\linewidth}{#1}} % Comando para crear reglas divisoras con un argumento para su altura